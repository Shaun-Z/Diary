\univlogo

{\Huge April 8}\vspace{5mm}

\section*{After-class assignments}

\subsection*{4.5}

Derive a gradient descent training rule for a single unit with output o, where

\begin{equation}
\begin{aligned}
    o=w_0+w_1x_1+w_1x_1^2+\dots+w_nx_n+w_nx_n^2
\end{aligned}
\end{equation}

Answer:
\begin{equation}
\begin{aligned}
    E(\vec{w})&=\cfrac{1}{2}\sum_{d\in D}(t_d-o_d)^2\\
    \cfrac{\partial E}{\partial w_i}&=\sum_{d\in D}(t_d-o_d)\cdot(x_i+x_i^2)\\
    \therefore \Delta w_i&=-\eta\cfrac{\partial E}{\partial w_i}=\eta\sum_{d\in D}(t_d-o_d)(-x_i-x_i^2)
\end{aligned}
\end{equation}

\subsection*{4.6}

Explain informally why the delta training rule in Equation(4.10) is only an approximation to the true gradient descent rule of Equation(4.7).

\begin{equation}
\begin{aligned}
    \Delta w_i&=\eta(t-o)x_i &(4.10)\\
    \Delta w_i&=\eta\sum_{d\in D}(t_d-o_d)x_{id} &(4.7)    \nonumber
\end{aligned}
\end{equation}

Answer:
Batch gradient descent uses the whole data set to determine the $\Delta w_i$, while stocastic gradient descent uses single random point.

Equation (4.10) corresponds to incremental gradient descent in which weights 
are updated after seeing each of the examples, while Equation (4.7) corresponds 
to true gradient descent in which the weights are updated after seeing all the 
examples once.